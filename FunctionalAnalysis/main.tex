% PROGRAMA DE PÓS-GRADUAÇÃO EM ECONOMIA APLICADA
% UNIVERSIDADE FEDERAL DO RIO GRANDE - FURG

% ===========================================================

\documentclass[a4paper, 11pt]{extarticle}

%%%%%%%%%%%%%%%%%%%%%%%%% PACOTES %%%%%%%%%%%%%%%%%%%%%%%%%%%%%%%%%%

\usepackage[margin =0.5in, bottom = 2in]{geometry}
\usepackage[utf8]{inputenc}
\usepackage[T1]{fontenc}
\usepackage{graphicx}
\usepackage{xcolor} 
\usepackage{enumitem} 
\usepackage{lipsum}	 
\usepackage{amsfonts,amsmath,amssymb, amsthm} 
\usepackage[colorlinks=true, linkcolor=blue, citecolor=blue, urlcolor=blue]{hyperref}
\usepackage{layout}
\usepackage{pifont, todonotes}
% Configuração da Fonte
\usepackage{times} % Fonte: Times New Roman
\usepackage{layout}
\usepackage{multicol}

% PACOTES ESSENCIAIS
\usepackage{caption}  
\usepackage{titlesec}

\titleformat{\section}
  {\normalfont\Large\bfseries} % formato de la fuente
  {\S \hspace{-0cm} \thesection}              % etiqueta de la sección (con símbolo ♦)
  {0.5em}                        % espacio entre el número y el título
  {}                           % código antes del título

%\setlength{\footskip}{0pt} % Espacio entre texto y pie de página
\renewcommand{\footnoterule}{\vspace{5pt}\hrule width 0.3\linewidth\vspace{5pt}}
\setlength{\textheight}{750pt}
\setlength{\textwidth}{370pt}
%\setlength{\footskip}{25pt}
\setlength{\marginparwidth}{145pt}
\setlength{\marginparsep}{12pt}
\setlength{\headsep}{10pt}
\setlength{\topmargin}{-60pt}
\definecolor{verdeoscuro}{RGB}{0,100,0} % Un verde bastante oscuro
\definecolor{rojoscuro}{RGB}{139,0,0} % rojo oscuro tipo "DarkRed"

\setlength{\parindent}{0pt}

\todostyle{orange}{
    linecolor=orange,        % Flecha negra
    bordercolor=orange,      % Borde negro
    textcolor=black!90,      % Borde negro
    backgroundcolor=white!33,  % Sin relleno de color
    size=\scriptsize           % Tamaño de texto
}
\todostyle{green}{
    linecolor=verdeoscuro,        % Flecha negra
    bordercolor=verdeoscuro,      % Borde negro
    textcolor=black!90,      % Borde negro
    backgroundcolor=white!33,  % Sin relleno de color
    size=\scriptsize           % Tamaño de texto
}
\todostyle{red}{
    linecolor=red,        % Flecha negra
    bordercolor=red,      % Borde negro
    textcolor=black!90,      % Borde negro
    backgroundcolor=white!33,  % Sin relleno de color
    size=\scriptsize           % Tamaño de texto
}
\todostyle{red2}{
  linecolor=rojoscuro,        % Flecha negra
  bordercolor=rojoscuro,      % Borde negro
  textcolor=black!90,      % Borde negro
  backgroundcolor=white!33,  % Sin relleno de color
  size=\scriptsize           % Tamaño de texto
}
\todostyle{gris}{
  linecolor=gray,        % Flecha negra
  bordercolor=gray,      % Borde negro
  textcolor=black!90,      % Borde negro
  backgroundcolor=white!33,  % Sin relleno de color
  size=\scriptsize           % Tamaño de texto
}

\begin{document} %Início do Documento
%\layout
%\newpage 
\- \vspace{-0.3in}
\par\noindent\includegraphics[width = 1.42\textwidth]{Imagens/Header.png}

%%%%%%%%%%%%%%%%%%%%%%%% CONFIG. DO TÍTULO %%%%%%%%%%%%%%%%%%%%%%%%
%%\vspace{n}

\begin{center} %Inicia a centralização
{ \hspace{5.7cm}\Large\textbf{Functional Analysis\ \ | \ \  2025-I}} \vspace{0.1in}   \\ {\small
\hspace{5.7cm}Nestor Heli Aponte Avila\(^1\) \\ 
\hspace{5.7cm}\href{mailto:n267452@dac.unicamp.br}{\url{n267452@dac.unicamp.br}}} %\\
%Prof. Piper Pimienta \\
%\href{mailto:pipe@dac.unicamp.br}{\url{pipe@dac.unicamp.br}}}

\end{center} %Finaliza a centralização

%%%%%%%%%%%%%%%%%%%%%%%%% PREAMBULO %%%%%%%%%%%%%%%%%%%%%%%%%%%%%%%%%%
\makeatletter
\renewcommand\@makefnmark{%
  \hbox{\@textsuperscript{\color{magenta}\normalfont\@thefnmark}}}
\renewcommand\@makefntext[1]{%
  \noindent\makebox[1.8em][r]{\textsuperscript{\color{magenta}\@thefnmark}}#1}
\makeatother

\theoremstyle{definition}
\newtheorem{definition}{\rotatebox{45}{\large \(\square\)}}[section]

\theoremstyle{plain}
\newtheorem{lemma}{{\scriptsize \(\square\)}}[section]
\newtheorem{proposition}{{\large \(\square\)}}[section]
\newtheorem{theorem}{{\large \(\blacksquare\)}}[section]

\makeatletter
% Eliminar la puntuación del estilo 'plain' y 'definition'
\def\@thm@headpunct{} % Elimina el punto de los títulos de los teoremas

% Modificar el estilo de 'plain' y 'definition'
\def\th@plain{%
  %\thm@headfont{\normalfont\bfseries} % Mantener la fuente normal, sin negrita
  %\thm@notefont{\normalfont} % El texto de la nota (si la hay) no será en negrita
  \thm@headpunct{} % Sin puntuación adicional
}

\def\th@definition{%
  %\thm@headfont{\normalfont\bfseries} % Fuente normal para el título
  %\thm@notefont{\normalfont} % El texto de la nota (si la hay) no será en negrita
  \thm@headpunct{} % Sin puntuación adicional
}

\def\th@remark{%
  %\thm@headfont{\normalfont\bfseries} % Fuente normal para el título
  %\thm@notefont{\normalfont} % El texto de la nota (si la hay) no será en negrita
  \thm@headpunct{} % Sin puntuación adicional
}
\makeatother

\definecolor{verdeoscuro}{RGB}{0,100,0} % Un verde bastante oscuro
\definecolor{rojoscuro}{RGB}{139,0,0} % rojo oscuro tipo "DarkRed"

\theoremstyle{remark}
\newtheorem*{example}{\textcolor{verdeoscuro}{\underline{{Example}}}}
\newtheorem*{exercise}{\textcolor{rojoscuro}{\underline{{Exercise}}}}
\newtheorem*{note}{\(*\)}

\newcommand{\R}{\mathbb{R}}
\newcommand{\N}{\mathbb{N}}
\newcommand{\C}{\mathbb{C}}
\newcommand{\K}{\mathbb{K}}
\newcommand{\Li}{\mathcal{L}}
\newcommand{\Ei}{}
\newcommand{\Ef}{}
\newcommand{\ab}{^{\text{\ding{71}}}}
\newcommand{\ce}{^{\text{\ding{70}}}}
\newcommand{\dual}[1]{{\textcolor{red}{#1}}}

%%%%%%%%%%%%%%%%%%%%%%%%%% REFERÊNCIAS %%%%%%%%%%%%%%%%%%%%%%%%%%%%%%%%

%\section{Espacios Vectoriales Normados}

\subsection{Definición y Ejemplos}

\begin{definition}
    Sea \(E\) un espacio vectorial sobre \(\K\). Una función \(\|\cdot\|: E\rightarrow \R^+ \) es \emph{norma} sii \(\forall x,y\in E \ \text{y}\ \alpha\in \K\),   
    \begin{itemize}
        \item \(\|x\|\geq 0\) e \(\|x\|=0 \Leftrightarrow x = 0\). 
        \item \(\|\alpha x\| = |\alpha|\|x\|\). 
        \item \(\|x+y\|\leq \|x\|+\|y\|\). 
    \end{itemize} 
\end{definition}

\begin{note}
    \((E,\|\cdot\|)\) espacio normado implica espacio métrico, vale la teoría existente para ellos en particular convergencia. 
\end{note}

\begin{definition}
    Sea \((x_n)\subset E\). Decimos que \(x_n\to x \in E\) sii \(\lim\limits_{n\to\infty} \|x_n - x\| = 0\).  
\end{definition}

\Ei

\begin{exercise}
    Las operaciones algebraícas en \(E\) son funciones continuas. 
\end{exercise}

\Ef

\begin{definition}
    \((E,\|\cdot\|)\) es \emph{Banach} sii es completo con la métrica inducida por la norma. 
\end{definition}

\Ei 

\begin{example}
    \((\R,\|\cdot\|_2)\) y \((\C,\|\cdot\|_2)\) son espacios de Banach. 
\end{example}

\Ef  

\begin{proposition}
    Sea \(E\) Banach y \(F\leq E\) un subespacio vectorial, entonces \(F\) es Banach sii \(F\) es cerrado en \(E\). 
\end{proposition}

\Ei 

\begin{example}
    Sea \(B(X):= \{f:X\rightarrow \K \ |\ f \ \text{es acotada}\}\) e \(\|f\|_\infty := \sup |f(x)|\), entonces \(B(X)\) es Banach. Sea \([a,b]\subset \R\), conjunto compacto, observe que \(C^0[a,b]\leq B[a,b]\) normado.   
\end{example}
\begin{exercise}
    Complete los detalles del ejemplo, y muestre que \(C^0[a,b]\) es Banach.
\end{exercise}
\begin{example}
    Considere \(C^1[a,b] \leq C^0[a,b]\), no es Banach. Sin embargo con \(\|f^{(1)}\|_{\infty^1}:= \|f\|_\infty + \|f^{(1)}\|_\infty\) si que lo es. En general \(C^k[a,b]\) es Banach con \(\|f^{(k)}\|_{\infty^k} = \sum_{i=0}^k \|f^{(i)}\|_\infty\). 
\end{example}

\Ef

\begin{proposition}
    Si \(B = \{x_1,\ldots,x_n\}\) es una base l.i. de \(E\), entonces \(\exists c>0, \forall a \in \K^n\) tal que \(\left\|\sum a_ix_i\right\|\geq c \sum |a_i|\). 
\end{proposition}
\begin{theorem}
    Todo espacio \(E\) tal que \(\dim{(E)}<\infty\) es Banach. Consecuentemente, también lo son todos los \(F\leq E\) cerrados. 
\end{theorem}

\Ei

\begin{example}
    Sea \(c_0=\{(a_k) \subset \K : a_k \rightarrow 0\}\) con las operaciones usuales y \(\|(a_k)\|_\infty := \sup |a_k|\). Así dado \(c_0\) es Banach. 
\end{example}
\begin{example}
    Sea \(c_{00}:= \{(a_k)\in c_0 : \exists n_0\in \N \text{ tal que } a_k = 0\text{ para } k\geq n_0\}\). No cerrado, no Banach. 
\end{example}

\Ef

\subsection{Espacios \(L_p,\ \ell_p\)}

\begin{definition}
    Sea \(\Li(X,\Sigma,\mu)\) espacio de medida. Consideremos \(\Li_p(X,\Sigma,\mu)\) el subespacio de las \(f:X\rightarrow\K\) tales que para \(1\leq p<\infty \) el valor \(\|f\|_p := \displaystyle \left(\int_X |f|^p\ d\mu\right)^{\frac{1}{p}}< \infty \). 
\end{definition}
\begin{theorem}
    \emph{Hölder.} Sean \(p,q>1\) tales que \(1/p + 1/q = 1\). Si \(f\in \Li_p\) e \(g\in \Li_q\) entonces \(fg\in \Li_1\) y \(\|fg\|_1\leq \|f\|_p\cdot\|g\|_q\).  
\end{theorem}
\begin{theorem}
    \emph{Minkowski.} Si \(f,g\in \Li_p\) entonces \(f+g\in \Li_p\) y \(\|f+g\|_p\leq \|f\|_p+\|g\|_p\) 
\end{theorem}
\begin{note}
    Note que \(\|f\|_p = 0 \nRightarrow f=0\). Ello motiva la siguiente consideración.  
\end{note}
\begin{definition}
    Sean \(f,g\in \Li_p\). Decimos que \(f\sim g\) si \(f=g\) \(\mu\)-casi siempre, es decir, \(\exists A\in \Sigma \) tal que \(\mu(A)=0\) y \(f(x)=g(x)\) para todo \(x\in X\setminus A\). 
\end{definition}
\begin{theorem}
    El conjunto \(L_p = \Li_p/\hspace{-0.1cm}\sim\) con las operaciones \([f]+[g] = [f+g]\), \([cf]=c[f]\) y norma \(\|[f]\|_p := \|f\|_p\) es Banach. 
\end{theorem}
\begin{definition}
    Sea \(\Li_\infty\) el espacio de las \(f\) medibles acotadas \(\mu\)-cuasi siempre\footnote{\(|f(x)|\leq k < \infty\) para cada \(x\in X\setminus N\) donde \(\mu(N\in \Sigma)=0\).}. Para cada \(x\in X\setminus N\) sean \(S_f(N) = \sup|f(x)|\) y \(\|f\|_\infty:= \inf S_f(N)\) de los \(N\in \Sigma\) tales que \(\mu(N) = 0\). 
\end{definition}
\begin{theorem}
    El espacio \(L_\infty = \Li_\infty/\hspace{-0.1cm}\sim \) con \(\|[f]\|_\infty := \|f\|_\infty\) es Banach. 
\end{theorem}

\Ei

\begin{example}
    Sean \(p\geq 1\) y \(\ell_p = \left\{(a_k)\subset \K : \displaystyle \sum |a_k|^p < \infty\right\}\). Considere \(\Sigma = \wp(\N)\) y \(\mu_c\) la medida de conteo. El espacio \(\ell_p\) coincide con \(L_p(\N,\wp(\N), \mu_c)\), en este caso las operaciones son simplemente las usuales de sucesiones y \(\displaystyle \|(a_n)\|_p = \left(\sum |a_k|^p\right)^{\frac{1}{p}}\). Entonces \(\ell_p\) es Banach.
\end{example}

\Ef

\begin{proposition}
   \emph{Hölder-Minkowski en sucesiones.} Para \(p,q>1\) tales que \(1/p+1/q = 1\) y \(\forall n\in \N\) vale que 
   \[\sum^n |a_ib_i|\leq \left(\sum^{n}|a_i|^p\right)^{\frac{1}{p}} \cdot \left(\sum^{n}|b_i|^q\right)^{\frac{1}{q}}\text{\ \ y \ \ } \left(\sum^{n}|a_i + b_i|^p\right)^{\frac{1}{p}}\leq \left(\sum^{n}|a_i|^{p}\right)^{\frac{1}{p}} + \left(\sum^{n}|b_i|^{p}\right)^{\frac{1}{p}}\]
\end{proposition}

\Ei

\begin{example}
    El espacio \(\ell_\infty = \{(a_k)\subset \K: \sup |a_k| <\infty\} \) con la norma \(\|(a_k)\|_\infty = \sup |a_k|\) es Banach. Esto es directo observando que \(\ell_\infty = L_\infty(\N,\wp(\N),\mu_c) = B(\N)\)\footnote{Verificando alguna de las dos igualdades.}.  
\end{example}

\Ef

\subsection{Compacidad}

\begin{definition}
    \(A\subseteq X\) es compacto sii todo cubrimiento abierto de \(A\) admite un subcubrimiento finito. 
\end{definition}
\begin{note}
    En espacios métricos vale decir que toda \((a_k)\subset A\) admite una subsucesión \((a_{k_i})\) tal que \(a_{k_i} \rightarrow a \in A\). 
\end{note}
\begin{proposition}
    Sea \((E,\|\cdot\|)\) con \(\dim(E)<\infty\), entonces los compactos de \(E\) son precisamente los cerrados y acotados. 
\end{proposition}
\begin{note}
    \emph{Colorario.} La \emph{bola unitaria} \(B_E := \{x\in E : \|x\|\leq 1 \}\) es compacta en espacios de dimensión finita. 
\end{note}
\begin{proposition}
    \emph{Riesz.} Si \(M<E\) cerrado y \(\theta \in (0,1)\), entonces \(\exists y\in E\setminus M,\forall x\in M\) tal que \(\|y\|=1\) y \(\|y-x\|\leq \theta \). 
\end{proposition}
\begin{theorem}
    \(\dim(E)<\infty\) sii \(B_E\) es compacta en \(E\). 
\end{theorem}

\vfill 

\subsection{Espacios Separables}

\begin{definition}
    \(E\) es separable sii \(\exists D\subset E\) enumerable y denso en \(E\). 
\end{definition}

\Ei

\begin{example}
    Espacios con \(\dim (E)<\infty\) son separables. Caso de \(\mathbb{Q}\subset \R\). 
\end{example}

\Ef

\begin{proposition}
    \(E\) es separable sii \(\exists A\subset E\) enumerable tal que \(\langle A\rangle\) es denso en \(E\). 
\end{proposition}

\Ei

\begin{example}
    \(c_0\) y \(\ell_p \) son separables, mientras \(\ell_\infty\) no lo es. 
\end{example}

\Ef

\begin{theorem}
    \emph{Aproximación de Weierstrass.} \(f:[a,b]\rightarrow \K\) continua \(\Rightarrow \forall\epsilon>0,\forall x\in [a,b], \exists P:\K\rightarrow \K \) tal que \(|P(x) - f(x)|<\epsilon \). 
\end{theorem}

\Ei

\begin{example}
    \(C[a,b]\) es separable. Hint:\(\langle t \rangle \). 
\end{example}
\begin{example}
    \(L_p[a,b]\) es separable. Hint: continuas y polinomios. 
\end{example}

\Ef

\begin{proposition}
    Si \(E\) es separable entonces \(F\leq E\) también lo es. 
\end{proposition}

%\section{Operadores Lineales}

\begin{definition}
    Un \emph{operador lineal continuo} es una función \(T:E\to F \) que verifica lo siguiente, 
    \begin{itemize}
        \item \(\forall x,y \in E,\forall \alpha \in \K \ \text{tenemos } T(\alpha x+y) =  \alpha T(x) + T(y)\). 
        \item \(\forall x_0 \in E,\forall \epsilon > 0, \exists \delta > 0\) tal que \(\|x-x_0\|<\delta \Rightarrow \|T(x)-T(x_0)\|<\epsilon\). 
    \end{itemize}
 \end{definition}
\newcommand{\Lc}{\mathcal{L}}
\begin{note}
    \(\Lc(E,F)=\{T:E\to F\ |\ T \text{ es lineal continuo}\}\) es espacio vectorial sobre \(\K\). Si \(F = \K\) entonces \(\Lc(E,\K)=E'\), el espacio dual, cuyos elementos son funciones.   
\end{note}
\begin{definition}
    \(E\cong F\) sii \(\exists T\in \Lc(E,F)\) biyectivo cuyo inverso \(T^{-1}\in \Lc(F,E)\). 
\end{definition}
\begin{definition}
    Una función \(f:E\to F \) es una \emph{isometría} sii \(\forall x\in E\) tenemos \(\|f(x)\| = \|x\|\).      
\end{definition}
\begin{note}
    Si \(f\) es lineal entonces es una \emph{isometría lineal}. Si \(f\) es un isomorfismo entonces le llamamos \emph{isomorfimo isométrico}. 
\end{note}

\Ei 

\begin{exercise}
    Toda isometría lineal es inyectiva y continua. 
\end{exercise}

\Ef

\subsection{Caracterización}

\begin{definition}
    Una función \(f:M\to N \) es \emph{Lipschitz} si \(\exists L>0, \forall x,y\in M\) tal que \(\|f(x)- f(y)\|\leq L \|x-y\|\).  
\end{definition}
\begin{definition}
    \(f\) es \emph{uniforme continua} si \(\forall \epsilon >0,\exists \delta >0,\forall x,y\in M\) tal que \(\|x-y\|<\delta \Rightarrow \|f(x)-f(y)\|<\epsilon\). 
\end{definition}
\begin{note}
    Lipschitz \(\Rightarrow\) uniforme continua \(\Rightarrow\) continua \(\Rightarrow\) continua en \(x_0\). 
\end{note}
\begin{theorem}
    Sea \(T\in \Lc(E,F)\). \(T\) es Lipschitz \(\Leftrightarrow\) \(T\) es uniforme continuo \(\Leftrightarrow\ T\) es continuo \(\Leftrightarrow\ \exists x_0\in E\) tal que \(T\) es continuo en \(x_0\) \(\Leftrightarrow\ T\) es continuo en \(0\)  \(\Leftrightarrow\ \sup\{\|T(x)\| : x\in B_E\}<\infty\) \(\Leftrightarrow \) \(\exists C\geq 0, \forall x\in E\) tal que \(\|T(x)\|\leq C \|x\|\). 
\end{theorem}
\begin{note}
    \emph{Colorario.} \(T\in \Lc(E,F)\) biyectivo es isomorfismo sii \(\exists C_1,C_2>0, \forall x\in E\) tal que \(C_1\|x\|\leq \|T(x)\| \leq C_2 \|x\|\). 
\end{note}
\begin{proposition}
    Sean \(E,F\) espacios normados, entonces \((a) \ \|T\| = \sup\{\|T(x)\|: x\in B_E\}\) es norma en \(\Lc(E,F)\); \((b) \ \forall T\in \Lc(E,F),\forall x\in E \) tenemos \( \|T(x)\| = \|T\|\cdot\|x\|\) y \((c) \ F\text{ Banach } \Rightarrow \Lc(E,F) \text{ Banach}\). 
\end{proposition}
\begin{note}
    \emph{Colorario (c).} \(E'\) es Banach. 
\end{note}

\Ei 

\begin{example}
    El operador identidad \(1_E: E \to E \in \Lc(E,E)\) para el cual \(\|1_E\|=1\); Operador nulo \(O:E\to F \in \Lc(E,F)\) que envía \(x\mapsto 0_F\) tenemos \(\|O\| = 0\). 
\end{example}
\begin{example}
    Sea \(\varphi \in E'\) e \(y\in F\). Sea \(\varphi \otimes y: x \mapsto \varphi(x)y \in \Lc(E,F)\) y tiene norma \(\|\varphi\|\cdot\|y\|\).  
\end{example}
\begin{example}
    Sea \((b_n) \in \ell_p\). Considere \(T:\ell_\infty \to \ell_p\) tal que \((a_n)\mapsto (a_nb_n)\) \footnote{\(T\) es llamado \emph{operador diagonal} por \((b_n)\).}. 
\end{example}
\begin{example}
    Sea \(g\in L_p[0,1]\). Como en el ejemplo anterior considere \(T:C[0,1] \to L_p[0,1]\), \( T(f) = fg\). El operador \(T\in \Lc(C[0,1],L_p[0,1])\) y es llamado \emph{operador multiplicación}. 
\end{example}
\begin{exercise}
    \(T\) lineal en \(E\) con \(\dim(E)<\infty \Rightarrow \ T \) continuo. En dimensión infinita no siempre es cierto. 
\end{exercise}
\begin{example}
    Sea \(\mathcal{P}[0,1]\subset C[0,1]\) con la norma \(\|\cdot\|_\infty \). El operador derivación es lineal, suponga continuo, entonces \(\exists C, \forall p \in \mathcal{P}[0,1] \) tal que \(\|T(p)\|_\infty \leq C\|p\|_\infty\). Sea \(f_n = t^n\), tenemos \(n = \|f_n^{'}\|_\infty = \| T(f_n)\|_\infty \leq C\|f_n\|_\infty = C \). 
\end{example}

\Ef

\subsection{Teorema Banach-Steinhaus}

\begin{theorem}
    \emph{Baire.} Sea \(M\) espacio métrico completo y \(\left(F\ce_n\right) \subseteq M\) tal que \(M=\bigcup F_n\). Entonces \(\exists n_0\in \N\) tal que \(\overset{\circ}{F_{n_0}} \neq \emptyset\).    
\end{theorem}
\begin{theorem}
    \emph{Banach-Steinhaus.} Sean \(E\) Banach, \(F\) espacio normado y \((T_i)\) una sucesión de operadores en \(\Li(E,F)\) tales que \(\forall x\in E,\exists C_x <\infty\) tal que \(\sup \|T_i(x)\| < C_x\). Entonces \(\sup \|T_i\| < \infty\).  
\end{theorem}
\begin{note}
    \emph{Colorario.} Sea \((T_n)\subset \Li(E,F)\). Si \(\forall x\in E\) la sucesión \((T_n(x)) \to y\in F\) entonces \(T(x) = \lim T_n(x) \in \Li(E,F)\).  
\end{note}

\Ei 

\begin{example}
    \((x,y)\mapsto \frac{xy}{x^2+y^2}\), \((0,0) \mapsto 0\) es continua en  \(\R^2\setminus \{0\}\), en aplicaciones \emph{bilineales} no existen cosas así. 
\end{example}

\Ef

\begin{definition}
    Sean \(E_1, E_2 \text{ y } F \) espacios vectoriales. Una aplicación \(B:E_1\times E_2 \to F\) es \emph{bilineal} sii \(\forall x_1\in E_1,\forall x_2\in E_2 \text{ fijos, los operadores }B(x_1,\cdot): E_2 \to F\) y \(B(\cdot,x_2): E_1\to F\) son lineales. 
\end{definition}

\begin{note}
    \emph{Colorario.} Si \(E_2\) es completo y \(B:E_1\times E_2 \to F \) es bilineal y continuo a trozos entonces \(B\in \Li(E_1\times E_2, F)\).    
\end{note}

\Ei

\begin{example}
    \(\forall n\in\N\) sea \(\varphi_n: c_{00} \ni (a_j) \mapsto na_n \in \K\). Es claro que \((\varphi_n)\subset (c_{00})'\) y \(\|\varphi_n\|=n\), aquí \(\forall x\in c_{00} \text{ se tiene }\sup \|\varphi_n(x)\|<\infty\), sin embargo, \(\sup\|\varphi_n\|=\infty\). 
\end{example}

\Ef

\subsection{Teorema de la Aplicación Abierta}

\begin{definition}
    Nos referimos a \(B_E(x_0;r)= \{x\in E: \|x-x_0\|<r\}\) como bola abierta en \(E\) centrada en \(x_0\) de radio \(r>0\). 
\end{definition}
\begin{proposition}
    Sean \(E\) Banach, \(F\) espacio normado y \(F\leftarrow E : T \in \Li(E,F)\). Si existieran \(R,r >0\) tales que \(\overline{T(B_E(0;R))}\supseteq B_F(0;r)\) entonces \(T(B_E(0;R)) \supseteq B_F\left(0;\frac{r}{2}\right)\).  
\end{proposition}
\begin{theorem}
    \emph{Aplicación Abierta.} Sean \(E\text{ e }F\) Banach. Si \(F \leftarrow E:\overset{\twoheadrightarrow}{T}\in \Li(E,F)\) entonces \(T\) es una aplicación abierta.  
\end{theorem}
\begin{note}
    \emph{Colorario.} En particular si \(T\) es una biyección entonces \(E\cong F\).  
\end{note}

\Ei

\begin{exercise}
    Muestre que  \(T: c_{00}\to c_{00}\) tal que \((a_n)\mapsto \left(\frac{a_n}{n}\right)\) es lineal, continuo y biyectivo. 
\end{exercise}
\begin{example}
    En el ejercicio anterior \(T^{-1}\) no es continuo.  
\end{example}
\begin{example}
    Todo subespacio \(F\ce\leq C[0,1]\) tal que \(\dim (F) = \infty \) tiene al menos una función \(f\) tal que \(f\notin C^1[0,1]\). Hint: Contradicción -- Aplicación Abierta -- Teorema de Riesz. 
\end{example}

\Ef

\begin{definition}
    Sean \(E \text{ e } F\) espacios normados y \(T:E\to F\) lineal. El \emph{gráfico} de \(T\) es el conjunto, 
    \[G(T) = \{(x,T(x)):x\in E\}\subseteq E\times F. \]
\end{definition}
\begin{theorem}
    \emph{Gráfico Cerrado.} Sean \(E \text{ e } F\) Banach y \(T:E\to F\) lineal. El operador \(T\) es continuo sii \(G(T)\) es cerrado en \(E\times F\). 
\end{theorem}

\Ei

\begin{exercise}
    Si \(T\) no es continuo una de las implicaciones en el Teorema del Gráfico Cerrado continua valiendo. 
\end{exercise}
\begin{example}
    Sean \(E\) Banach y \(T:E\to E'\) lineal \emph{símetrico}, es decir, \(\forall x,y\in E \) tenemos \(T(x)(y)=T(y)(x)\). El operador \(T\) es continuo. Hint: Gráfico Cerrado. 
\end{example}

\Ef
%\section{Teoremas de Hahn-Banach}

\begin{proposition}
    \emph{Lemma de Zorn.} Todo conjunto parcialmente ordenado, no vacío y en el cual todo subconjunto totalmente ordenado tiene cota superior, tiene un elemento máximal. 
\end{proposition}

\Ei

\begin{exercise}
    Lemma de Zorn \(\Leftrightarrow \) Axioma de Elección. 
\end{exercise}

\Ef

%\begin{theorem}
%    \emph{Hanh-Banach (en \(\R\)).} Sean \(E\) un espacio (sobre \(\R\)) normado y \(p:E\to \R\) una función tal que, 
%    \begin{itemize}
%        \item \(\forall a>0,\forall x\in E\) se tiene \(p(ax) = ap(x)\). 
%        \item \(\forall x,y\in E\) su cumple \(p(x+y) \leq p(x)+p(y)\). 
%    \end{itemize} 
%    Sean tambien \(G\leq E \) y \(\varphi : G \to \R\) un operador lineal tal que \(\forall x\in G \text{ se tiene }\varphi(x)\leq p(x)\). Entonces \(\exists \overset{\sim }{\varphi} : E \to \R\) lineal que extiende a \(\varphi\), es decir, \(\overset{\sim }{\varphi}(x)\Big|_{G} = \varphi(x) \) y que además satisface \(\forall x\in E \text{ que }\overset{\sim}{\varphi}(x)\leq p(x)\). 
%\end{theorem}
\begin{theorem}
    \emph{Hanh-Banach (en \(\K\)).} Sean \(E\) un espacio (sobre \(\K\)) normado y \(p:E\to \R\) una función tal que, 
    \begin{itemize}
        \item \(\forall a>0,\forall x\in E\) se tiene \(p(ax) = |a|p(x)\). 
        \item \(\forall x,y\in E\) su cumple \(p(x+y) \leq p(x)+p(y)\). 
    \end{itemize} 
    Si \(G\leq E \) y \(\varphi : G \to \K\) es un operador lineal tal que \(\forall x\in G \text{ se tiene }|\varphi(x)|\leq p(x)\), entonces \(\exists \overset{\sim }{\varphi} : E \to \K\) lineal que extiende a \(\varphi\), es decir, \(\overset{\sim }{\varphi}(x)\Big|_{G} = \varphi(x) \) y que además satisface \(\forall x\in E \text{ que }|\overset{\sim}{\varphi}(x)|\leq p(x)\). 
\end{theorem}
\begin{note}
    \emph{Colorario(s).} \begin{itemize}
        \item Si \(\varphi \) es continuo entonces \(\overset{\sim}{\varphi}\) también y \(\|\varphi\| = \|\overset{\sim}{\varphi}\|\). 
        \item Si \(E\) es un espacio normado entonces \(\forall x_0\in E \setminus \{0\},\exists \varphi\in E'\) tal que \(\|\varphi\| = 1\) y \(\varphi(x_0) = \|x_0\|\). 
        \item Si \(E\neq \{0\}\) y \(x\in E \) entonces \(\|x\| = \sup\{ |\varphi(x)|: \varphi \in E' \text{ y } \|x\|\in B_E\} \) cuyo valor alcanza.   
    \end{itemize}
\end{note}

\subsection{Versiones Vectoriales del Teorema de Hahn-Banach}

\begin{definition}
    Sea \(E\) Banach y \(P\in \Li(E,E)\) es una \emph{proyección} sii \(P^2 = P \circ P = P \). 
\end{definition}
\begin{note}
    Si \(P\neq 0\) entonces \(\|P\|\geq 1 \). 
\end{note}
\begin{proposition}
    Sea \(F\leq E\). Entonces son equivalentes \((a)\ \exists P\in \Li(E,E)\) proyección tal que \(P(E) = F   \) e \((b) \ F\ce \leq E \) e \(\exists G\ce \leq E  \) tal que \(E = F\oplus G\). 
\end{proposition}
\begin{note}
    \(F = \{x\in E : P(x) = x\}\) e \(G = \ker(P)\). 
\end{note}
\begin{definition}
    \(F\leq E \) es complementado si satisface alguna \((a)\) o \((b)\). Es \(\lambda\)-complementado si \(\|P\| = \lambda\). 
\end{definition}

\Ei

\begin{example}
    Todo \(F\leq E\) con \(\dim (F)<\infty \) es complementado. Hint: Base + $a_j$ + Hahn-Banach.   
\end{example}
\begin{example}
    Siendo \(E\) y \(F\) Banach. La proyección \(E\times F \ni (x,y)\mapsto (x,0) \in E\times \{0\}\) deja ver que \(E\) es $1$-complementado. 
\end{example}
\begin{example}
    SI existen espacios cerrados que son no complementados, créditos a Murray \((\ell_p)\) y Phillips \((\ell_\infty)\). 
\end{example}

\Ef

\begin{proposition}
    Sean \(G\) Banach, \(F\leq E \) complementado e \(T \in \Li(F,G)\), entonces \(\exists \overset{\sim}{T} \in \Li(E,G)\). 
\end{proposition}
\begin{proposition}
    Si \(F\leq E\) no complementado, entonces \(\not\exists T\in \Li(E,F), \forall x\in F \) tal que \(T(x) = x \).  
\end{proposition}
\begin{note}
    La identidad no puede ser extendida continuamente a \(E\). 
\end{note}
\begin{theorem}
    \emph{Phillips.} Sean \(F\leq E\) e \(T\in \Li(E,\ell_\infty)\). Entonces \(\exists \overset{\sim}{T}\in \Li(E, \ell_\infty)\), con \(\|\overset{\sim}{T}\| = \|T\|\). 
\end{theorem}
\begin{note}
   \emph{Colorario.}  Si \((\ell_\infty)\ce \leq E\) entonces \(\ell_\infty\) es $1$-complementado en \(E\). 
\end{note}

\subsection{Aplicaciones de Hahn-Banach a Espacios Separables}

\begin{proposition}
    Sean \(M\ce \leq E\), \(y_0\in  E\setminus M\) y \(d = \textnormal{dist}(y_o,M)\). Entonces \(\exists \varphi \in E', \forall x\in M\) tal que \( \|\varphi\|=1 \), \(\varphi(y_0) = d\) y \(\varphi(x)=0\). 
\end{proposition}
\begin{theorem}
    Si \(E'\) es separable, entonces \(E\) también. 
\end{theorem}
\begin{proposition}
    \(\forall E\) separable se tiene \(E\cong F\leq \ell_\infty\) isómetricamente.  
\end{proposition}

\subsection{Formas geométricas del Teorema de Hahn-Banach (\(\K = \R\)) }

\begin{definition}
    Sea \((V,+,\cdot)\neq \{0\}\). El subespacio \(W< V \) es \emph{hiperplano} sii \(W < W_1< V\) implica \(W_1= V\). 
\end{definition}
\begin{proposition}
    \(W<V\) es hiperplano sii \(\exists \varphi \neq 0: V \to \R \) lineal tal que \(W = \ker(\varphi)\). 
\end{proposition}
\begin{note}
    Si \(H<V\) es un hiperplano entonces \(v_0 + H = \{v\in V: \varphi(v) = \varphi(v_0) = a\in \R\}\) es un \emph{hiperplano afín}. 
\end{note}
\begin{proposition}
    \(H\ce <V\) sii \(\varphi \) es continua. 
\end{proposition}
\begin{definition}
    Sea \(C\ab \ni 0\subseteq E\) convexo. Llamamos \emph{funcional de Minkowski} a la aplicación \(p_C:E\to \R\) que envía \(x\mapsto \inf \left\{a>0: \frac{x}{a} \in C \right\}\). 
\end{definition}
\begin{proposition}
    El funcional de Minkowski verifica \(\forall b>0, \forall x,y\in E\) que:  \((a)\ p_C(bx) = bp_C(x)\); \((b) \ C = \{x\in E : p_C(x)<1\}\); \((c) \ \exists M>0 \text{ tal que } 0\leq p_C(x)\leq M \|x\|\) e \((d)\ p_C(x+y) \leq p_C(x)+p_C(y)\).  
\end{proposition}
\begin{proposition}
    Sean \(\emptyset \neq C\ab \subset E \) conexo e \(x_0\in E\setminus C\), entonces \(\exists \varphi \in E', \forall x\in C\) tal que \(\varphi(x)<\varphi(x_0)\).
\end{proposition}
\begin{proposition}
    Sean \(0\neq \varphi \in E'\) y \(\emptyset \neq A\ab  \subset E\) convexo, entonces \(\varphi(A)\ab\subseteq \R\).  
\end{proposition}
\begin{theorem}
    \emph{Primera Forma Geométrica del Teorema de Hahn Banach.} Sean \(\emptyset \neq A,B \subset E\) disjuntos. Si \(A\ab\subset E\) entonces \(\exists \varphi \in E', \exists a \in \R, \forall x\in A, \forall y\in B\) se tiene \(\varphi(x) < a \leq \varphi(y)\). 
\end{theorem}
\begin{note}
    En este caso decimos que el hiperplano \([\varphi =  a]\) separa a \(A\) de \(B\). 
\end{note}
\begin{theorem}
    \emph{Segunda Forma Geométrica del Teorema de Hahn Banach.} Sean \(\emptyset \neq A,B \subset E\) disjuntos. Si \(A\ce\subset E\) y \(B\) es compacto entonces \(\exists \varphi \in E', \exists a,b \in \R, \forall x\in A, \forall y\in B\) se tiene \(\varphi(x) < a <b \leq \varphi(y)\). 
\end{theorem}
\begin{note}
    Ahora decimos que \(\forall c\in (a,b)\) el hiperplano \([\varphi = c]\) separa estrictamente a \(A\) de \(B\). 
\end{note}
\begin{note}
    \emph{Colorario.} Si \(M\ce\leq E\) entonces \(\forall x_0\in E\setminus M,\exists \varphi \in E', \forall x\in M\) se tiene \(\varphi(x_0)=1\) y \(\varphi(x) = 0\). 
\end{note}
\begin{note}
    \emph{Colorario.} Sea \(x_0\in M\leq E\), entonces \(x_0\in \overline{M}\) sii \(\forall \varphi \in E', \forall x\in M\) se tiene \(\varphi(x_0) = 0\) siempre que \(\varphi(x)=0\). 
\end{note}
\section{Dualidad y Espacios Reflexivos}

\subsection{Duales \(L_p(X,\Sigma, \mu)\)}
\-\vspace{-0.25in}

\Ei

%\begin{example}
%    Sea \(\dual{g \in L_{p^*}}\). Considere \(\varphi_g: L_{p} \to \K \) tal que \(f\mapsto \int fg \ d\mu\). Por Holder 
%    \[|\varphi_g(f)| =\Big|\int fg \Big| \leq \int |fg| \leq \|g\|_\dual{p*} \cdot \|f\|_p = C \cdot \|f\|_p.\] 
%    Con esto, \(\varphi_g\) es bien definida, \(\varphi_g\in L_p^{'}\) y \(\|\varphi_g\|\leq \|g\|_\dual{p^*}\)
%\end{example}
\- \todo[gris]{Linealidad inmediata}\todo[gris]{Continuidad, cálcule \(|\varphi_g(f)|\) aplique Hölder \(\Rightarrow \varphi_g \in (L_p){'}\) y \(\|\varphi_g\|\leq \|g\|_{p^*}\)}\todo[gris]{\(\|g\|_{p^*}\leq \|\varphi_g\|\) y \(g\neq 0\). Si \(p>1\) tome \vspace{-0.2cm}\[f(x)=\frac{|g(x)|^{p^*-1}\overline{g(x)}}{\|g\|_{p^*}^{p^*-1}|g(x)|}\]\vspace{-0.2cm}}\todo[gris]{Sobreyectividad? Bien, gracias}
\begin{theorem}
    \(L_{p*} \simeq (L_{p}){'}\) isométricamente (para \(p=1\) suponemos \(p^*=\infty\) e \(\mu\) médida \(\sigma\)-finita) con la relación de dualidad dada por
    \[L_{p^*} \ni \textcolor{red}{g \mapsto \ } \underset{\underset{(L_p){'}}{\rotatebox{-90}{$\in$}}}{\textcolor{red}{\varphi_g}}: \overset{ \overset{L_p}{ \rotatebox{90}{$\in$}} }{f}\mapsto  \int_X fg\ d\mu \  \in \K.\]
\end{theorem}

\begin{note}
    El caso \(p=1\) no admite resumen, véase \cite[pág. 70]{botelho2025introduction}. Por lo pronto damos por sentado que \( L_\infty \simeq (L_1){'}\) isométricamente. 
\end{note}
\begin{proposition}
    \(\ell_{p^*} \simeq (\ell_{p}){'}\)\todo[orange, noline]{Caso particular \(L_p(\N, \wp(\N), \mu_c)\), solo busca exhibir como es la relación de dualidad en \(\ell_p\)}\todo[red,noline]{\(\ell_\infty\simeq (\ell_1){'}\), pero \( \ell_1 \not\simeq(\ell_\infty){'}\), pues sabemos que \(\ell_\infty \) no es separable} isométricamente (para \(p=1\) suponemos \(p^*=\infty\)) con la relación de dualidad 
    \[\ell_{p^*} \ni \textcolor{red}{(b_j) \mapsto \ } \underset{\underset{(\ell_p){'}}{\rotatebox{-90}{$\in$}}}{\textcolor{red}{\varphi_b}}: \overset{ \overset{\ell_p}{ \rotatebox{90}{$\in$}} }{(a_j)}\mapsto  \sum a_jb_j \  \in \K.\] 
\end{proposition}
\- \vspace{0.15cm} \todo[gris]{Bien definida, lineal, continua y \(\|\varphi_b\|\leq \|b\|_1\) es todo calcado}\todo[gris]{Sobreyectividad y \(\|b\|_1\leq \|\varphi\|\). Para \(\varphi \in (c_0){'}\) considere \((b_j) = (\varphi(e_j))\), defina \((\alpha_j)\)  \vspace{-0.2cm} \[\alpha_j=\frac{\overline{\varphi(e_j)}}{|\varphi(e_j)|}, \ \varphi(e_j)\neq 0 \text{ y }j\leq n\] Con eso consigue, para ver que \(\varphi_b = \varphi\) acuerdese que \(c_0 \ni a = \lim \sum a_je_j\)}
\begin{proposition}
    \(\ell_1\simeq (c_0)^{'}\) isométricamente, con la relación de dualidad dada por  
    \[ b \in \ell_1 \mapsto \varphi_b \in (c_0)^{'},  \hspace{0.5cm} \varphi_b((a_n)) = \sum a_jb_j\]
\end{proposition}
\- \vspace{1.2cm}
\subsection{Bidual y Espacios Reflexivos}

\- \vspace{-0.25in}\newline 
\- \todo[gris]{Linealidad y continuidad son inmediatos}\todo[gris]{Isometría, cálcule \(\|J_E(x)\|\), desarrolle y aplique el último colorario de H-B}
\begin{proposition}
    Para cada espacio normado \(E\), el operador lineal \(J_E:E\to E{''}\) que envía 
    \(E\ni x\mapsto \underset{\underset{E{'}}{\rotatebox{90}{$\in$}}}{\varphi}(x) \in \K \) es una isometría lineal llamada \emph{inmersión canónica} de \(E\) en \(E{''}\).  
\end{proposition}
\begin{note}
    Isometría lineal es inyectiva, luego, \(E{''}\) contiene una copia isometríca de \(E\).
\end{note}
\-  \todo[gris]{Tome \(\widehat{E} = \overline{J_E(E)}\subseteq E{''}\)}\todo[orange, noline]{Demostración aparentemente sencilla, pero recuerde que requirio H-B} 
\begin{proposition}
   Todo espacio normado \(E\) admite completación, esto es, \(\exists \widehat{E}\) Banach que contiene una copia isometríca de \(E\) densa en \(\widehat{E}\). 
\end{proposition}


\begin{definition}
    Un espacio normado \(E\) se dice \emph{reflexivo} si \(J_E: E \twoheadrightarrow E{''}\), en cuyo caso \(E\simeq E{''}\).
\end{definition}

\-  \todo[gris]{Directo, recuerde que \(E{'} \) es Banach}
\begin{proposition}
    Todo espacio refléxivo es Banach. 
\end{proposition}
\begin{example}
    Si \(\dim(E) =n <\infty\) entonces \(E\) es reflexivo. \textcolor{gray}{\(\rightarrow  \dim(E{''}) = n\)}% \Rightarrow J_E: E \twoheadrightarrow E^{\prime \prime}\)}
\end{example}
\begin{example}
    \(c_{00}\) no es reflexivo. \textcolor{gray}{\(\rightarrow\) Contrarrecíproca de la última proposición}% (c_0^\prime)^\prime = \ell_1^\prime = \ell_\infty\), es claro que \(c_0^{\prime\prime}\not\twoheadrightarrow \ell_\infty\)}
\end{example}

\begin{example}
    \(c_0\) no es reflexivo. \textcolor{gray}{\(\rightarrow\) Despliegue el bidual} 
\end{example}
\-  \todo[gris]{Directo, \(F{'} \) separable \(\Rightarrow F \) separable}
\begin{proposition}
    Si \(E\) es separable y reflexivo, entonces \(E{'}\) es separable.
\end{proposition}
\begin{example}
    \(\ell_1\) no es reflexivo. \textcolor{gray}{\(\rightarrow\) Inmediato de la proposición anterior}
\end{example}
\begin{definition}
    Sean \(E,F\) espacios normados e \(T \in \Li(E,F)\). El operador \(T{'} : F{'} \to E{'} \) que envía \(\varphi \mapsto \varphi (T(x))\) es llamado \emph{operador adjunto} de \(T\). 
\end{definition}
\begin{example}
    Sea \(T\in \Li(\ell_p,\ell_p)\) que envía \((a_1, a_2, \ldots) \mapsto (a_2, a_3, \ldots)\), \emph{backward shift operator} para los amigos, tiene como adjunto \(T{'} \in \Li(\ell_{p^*}, \ell_{p^*})\) que envía \((b_1, b_2, \ldots ) \mapsto (0, b_1, b_2, \ldots )\), bien llamado \emph{forward shift operator}. \textcolor{gray}{\(\rightarrow\) Desarrolle \((\varphi_b)(T(x))\)}  
\end{example}
\-  \todo[gris]{Linealidad es inmediata} \todo[gris]{Continuidad, desarrolle \(\|T{'}\|\), de aqui también sale que \(\|T{'}\|\leq \|T\|\)} \todo[gris]{\(\|T\|\leq \|T{'}\|\) se consigue aplicando el colorario de H-B a \(\|T(x)\|\)} \todo[gris]{\(+\) Isomorfismo. Para ver que \(T{'}\) es sobreyectivo, use \(T^{-1}\), inyectividad se consigue \(\ker(T{'})\)} \todo[gris]{\(+\) Isometría. Desarrolle \(\|T{'}(\varphi) \|\), tenga presente que \(x\in B_E \Leftrightarrow T(x) \in B_F\)}
\begin{proposition}
    Para cada \(T\in \Li(E,F) \) tenemos \(T{'} \in \Li(F{'}, E{'})\) y \( \|T\| = \|T{'}\|\). Más aun, si \(T\) es isomorfismo (isométrico) entonces \(T{'} \) también es isomorfismo (isométrico). 
\end{proposition}
\vspace{1.42in}
\-  \todo[gris]{Tome los isomorfismos isométricos \(T:L_{p^*}\to (L_p){'} \), \(S: L_p\to (L_{p^*}){'} \) e \((T^{-1}){'}\)}\todo[gris]{Muestre que \((T^{-1}){'} \circ S = J_{L_p}: L_p\to (L_p){'}\). Tome \(f\in L_p\), \(\varphi \in (L_p){'}\) e defina \(g = T^{-1}(\varphi) \in L_{p^*}\). Desarrolle \(((T^{-1}){'} \circ S)(f)(\varphi)\)}
\begin{proposition}
    Para \(1<p<\infty\) los espacios \(L_p\) son reflexivos. 
\end{proposition}

\vspace{0.75in}
\-  \todo[gris]{\((\Rightarrow)\) Tome \(\zeta{'''}\in E{'''}\), trabaje con \(J_{E{'}}\) e \((J_E){'}\) para hallar la preimagen}\todo[gris]{\((\Leftarrow)\) Supongo que no es reflexivo, y contradiga la existencia de \(0 \neq \varphi \in E{'''}\) que separe \(J_E(E)\) en \(E{''}\) (Prop. Aplicaciones de H-B)}
\begin{proposition}
    Un espacio \(E\) Banach es reflexivo sii \(E{'} \) es reflexivo.  
\end{proposition}
\vspace{0.46in}

\begin{example}
    \(\ell_\infty\) no es reflexivo. \textcolor{gray}{\(\rightarrow\) \(\ell_1\) no es reflexivo}
\end{example}
\section{Espacios de Hilbert}

\subsection{Espacios con Producto Interno}

\noindent 
\begin{definition}
   Sea \(E\) espacio vectorial sobre \(\K \). Un \emph{producto interno} sobre \(E\) es una función \(\langle \cdot , \cdot \rangle: E\times E \to \K \) tal que  \(\forall x, x{'}, y \in E, \ \forall \lambda \in \K\) se tiene,   \noindent
   \begin{enumerate}[label=(P\arabic*), leftmargin = 1.4cm]
    \item \(\langle x + x{'}, y \rangle = \langle x, y \rangle + \langle x{'} , y \rangle\).
    \item \(\langle \lambda x, y \rangle = \lambda\langle x, y \rangle \).
    \item \(\langle x, y \rangle = \overline{\langle y,x \rangle} \).
    \item Si \(x\neq 0\) entonces \(\langle x, x \rangle \in \R^{+}  \).
   \end{enumerate}
\end{definition}

\begin{lemma}
    \textcolor{rojoscuro}{\textbf{\underline{Exercise}}}\hspace{0.04cm} Demuestre que \((a)\ \langle x, 0 \rangle = \langle 0, y \rangle=0\); \((b)\ \langle x, x \rangle =0\) sii \(x=0\); \((c) \ \langle x, y + y{'} \rangle = \langle x, y \rangle + \langle x, y{'}\rangle\); \( (d)\ \langle x, \lambda y  \rangle = \overline{\lambda} \langle x, y \rangle\); \((e)\) \ Si \(\forall z \in E\) tenemos  \(\langle z, y \rangle = \langle z, y{'} \rangle\) entonces \(y = y{'}\). \textcolor{gray}{\(\rightarrow\) Consecuencia de las propiedades en la definición}
\end{lemma}

\- \todo[gris]{Suponga \(x, y \neq 0\). Tome \(a = \langle y,y\rangle \), \(b = \langle x,y\rangle\) y desarrolle el término 
\- \vspace{-0.1cm}
\[0 \leq \langle ax -by , ax - by\rangle \]
\- \vspace{-0.4cm}
} \todo[gris]{Para la otra afirmación suponga la igualdad y vea que pasa}
\begin{proposition}[Desigualdad de Cauchy-Schwarz] 
    Sean \(E\) espacio con producto interno e \(\|\cdot \|: x \mapsto \|x\| = \sqrt{\langle x , x \rangle}\). Entonces, \(\forall x,y\in E\) tenemos \(|\langle x,y\rangle| \leq \|x\| \|y\|\), igualdad cuando \(\exists \alpha \in \K\) tal que \(y = \alpha x\) (son l.d.).      
\end{proposition}
\-
\begin{note}
    \emph{Colorario.} La función \(\|\cdot \| : E \times E \to \R^+\) define una norma en \(E\). \textcolor{gray}{\(\rightarrow\) (N1) y (N2) son directas. Para (N3) desarme \(\|x+y\|^2\) y aplique Cauchy-Schwarz }
\end{note}

\begin{example}
   Sean \((x_n), (y_n) \in \ell_2\), la función que envía \(\langle (x_n), (y_n)\rangle \mapsto \sum a_n \overline{b_n}\) define un producto interno en \(\ell_2\) cuya norma inducida de hecho coincide con \(\|\cdot \|_2\). %\footnote{La convergencia de la serie de nuevo es garantizada por Hölder. } 
\end{example}
\begin{example}
    En \(L_2\) pasa exactamente lo mismo definiendo \(\displaystyle\langle f,g\rangle \mapsto \int_X f \ \overline{g}\ d\mu\). 
\end{example}

\-  \todo[gris]{Tome \(x_0 \in E\) fijo, agrupe la diferencia de las imagenes y aplique Cauchy-Schwarz}\todo[orange, noline]{Este detalle es importante pues nos garantiza compatibilidad topologica. }
\begin{proposition}
    Sean \(E\) espacio con producto interno e \(y_0\in E\) fijo. Las funciones \(x\mapsto \langle x, y_0\rangle \) e \( x \mapsto \langle y_0, x\rangle \) son continuas. 
\end{proposition}
\begin{definition}\todo[orange, noline]{Hilbert \(\Rightarrow\) Banach. }
    Un espacio con producto interno, completo con la norma inducida es llamado \emph{espacio de Hilbert}. 
\end{definition}
\begin{example}
    \(L_2\) e \(\ell_2\) son espacios de Hilbert. 
\end{example}
\- \\ 
\- \vspace{-0.12in}  \todo[gris]{i. Desarme con producto interno y simplifique la suma} \todo[gris]{ii. Misma cosa} \todo[gris]{iii. Haga un ejercicio similar con los términos que incluyen \(iy\), junte los sumandos y simplifique }
\begin{proposition}
    Sea \(E\) espacio con producto interno, entonces \(\forall x, y \in E\) se cumplen  
    \begin{enumerate}[label = \roman*., leftmargin = 1cm]
        \item Ley del Paralelogramo \(\rightarrow \ \ \|x+y\|^2 + \|x-y\|^2 = 2 (\|x\|^2 + \|y\|^2)\).
        \item Polarización (en \(\R\)) \( \rightarrow \ \ \langle  x, y \rangle = \frac{1}{4} (\|x+y\|^2 - \|x-y\|^2 )\).
        \item Polarización (en \(\C\))  \(\rightarrow \ \  \langle  x, y \rangle = \text{ii.}+ \frac{i}{4}(\|x+iy\|^2 - \|x-iy\|^2).\) 
    \end{enumerate}
\end{proposition}

\subsection{Ortogonalidad}

\begin{definition}
    Dos vectores \(x,y \in E\) espacio con producto interno, se dicen \emph{ortogonales} si \(\langle x, y \rangle =0 \), en cuyo caso denotamos \(x \perp y\). 
\end{definition}

\begin{lemma}[]
    \textcolor{rojoscuro}{\underline{Exercise}} (Teorema de Pitágoras) Si \(x\perp y\), entonces \(\|x+y\|^2 = \|x\|^2 + \|y\|^2\). \textcolor{gray}{\(\rightarrow\) Es cosa de reescribir con producto interno, y ver los términos que se anulan } 
\end{lemma}

\- \todo[gris]{Aplique la \(\epsilon\)-propiedad al \(\inf\) planteado con \(\epsilon = \frac{1}{n}\), de esto \(\exists (y_n) \subseteq M\)} \todo[gris]{Use (i.) en los índices \(n,m\), desarrolle los términos apuntando a mostrar que \((y_n)\) es Cauchy e \(y_n \to p \in M\)} \todo[gris]{Para la unicidad suponga un \(q\in M\) que cumple lo mismo e use (i.) para argumentar \(\|p-q\| = 0 \)}
\begin{theorem}
    Sean \(E\) espacio con producto interno e \(M \leq E\) Banach. Entonces, \(\forall x \in E, \exists ! p \in M \) tal que \(\|x-p\| = \text{dist}(x,M)\). 
\end{theorem}
\- \vspace{0.54in}
\begin{definition} \todo[orange, noline ]{La notación \(A^\perp \) puede parecer en conflicto con \(M^\perp\) del \emph{aniquilador} de \(M\), pero es la misma cosa}
    Sea \(A\subset E\) espacio con producto interno, llamamos \emph{complemento ortogonal} de A al conjunto \( A ^\perp : \{y \in E: \forall x \in A, \  \langle x,y \rangle = 0 \}\). 
\end{definition}

\begin{lemma}
    \textcolor{rojoscuro}{\underline{Exercise}} El complemento ortogonal verifica \((a) \  A \subseteq (A^\perp)^\perp  \), \( E^\perp  = \{0\}\) y \(\{0\}^\perp = E\); \((b) \ (A^\perp)\ce \leq E \); \((c) \ A \cap A^\perp = \{0\}\) si \( 0 \in A\), vacío en otro caso. \textcolor{gray}{\(\rightarrow \ (a)\) es directa; \((b)\) recuerde que el producto interno es continuo por coordenadas; \((c)\) solo describa el conjunto}
\end{lemma}
 
\-  \todo[gris]{(a) Tome \(p\) como en Teorema previo e muestre que \(x-p = q \in M^\perp\), tomando \(p + \lambda y \in M\) y desarrollando  
\- \vspace{-0.2cm} 
\[\|q\|^2 \leq \| x - (p+\lambda y )\|^2\]
\- \vspace{-0.5cm} \\ 
La unicidad es quasi inmediata 
} \todo[gris]{(b) La primera parte es inmediata de (a), para desigualdad restante de las normas, recuerde que \(H \ni x = p + q \) donde \(p\perp q\), pitágoras hace el resto }
\todo[gris]{(c) Es inmediato }
\todo[red, noline]{Es preciso en (a) que \(M\) sea completo, piense en \([e_j] \leq \ell_2 \)}
\begin{theorem}
    Sean \(H\) Hilbert e \(M\ce \leq H\), entonces 
    \begin{enumerate}[label = (\alph*), leftmargin = 1.1cm]
        \item \(H = M \oplus M^\perp \), es decir, \(\forall x \in H\) existen únicos \(p \in M\) e \(q\in  M^\perp\), tales que  \(x = p+q\), más aún, \(\|x-p\| = \text{dist}(x,M)\). %\footnote{El vector \(p\) es llamado \emph{proyección ortogonal} de \(x\) en \(M\). }
        \item \(P, Q: H \to H\) tales que \(P(x) = p\) e \(Q(x) = q \) son proyecciones, \(P(H)= M \) e \(Q(H) = M^\perp \). También \(\|P\|= \|Q\|=1\) si \(M < H\). %\footnote{El operador \(P\) es llamado \emph{proyección ortogonal} de \(H\) en \(M\). }
        \item \(P \circ Q = Q \circ P = 0\). 
    \end{enumerate}
\end{theorem}
\- \vspace{0.55cm}
\begin{note}[Colorario]
    En \(H\) Hilbert, todo \(M\ce < H\) no trivial es \(1\)-complementado. 
\end{note}

\subsection{Conjuntos Ortonormales en Espacios de Hilbert}

\begin{definition}
   Sea \(E\) espacio con producto interno.\todo[orange, noline]{Si \(S^\perp = \{0\}\), entonces es un \emph{sistema ortonormal completo}} Un conjunto \(S=\{x_i\}\subseteq E\) es un \emph{sistema ortonormal} si \(\forall i,j\) se tiene \(\langle x_i, x_j\rangle = \delta_{ij}\).  
\end{definition}

\begin{example}
    \(\{e_j: j\leq n < \infty\}\subseteq \K^n\) es sistema ortonormal completo.  
\end{example}
\begin{example}
    \(\{e_n: n\in \N\}\subseteq \ell_2 \) es sistema ortonormal completo. 
\end{example}

\-  \todo[gris]{(a) \(x = p+q \), represente \(p\) en la base de \(M\) y cálcule \(\langle x - p, x_j\rangle\)} \todo[gris]{(b) Desarme la expresión 
\- \vspace{-0.2cm}
\[\left\langle x - \sum \langle x, x_j\rangle x_j, x - \sum \langle x, x_j\rangle x_j\right\rangle\]
\- \vspace{-0cm} } \todo[orange, noline]{Solemos referirnos a esa suma en (a) como la \emph{mejor aproximación} de \(x\in M\)}
\begin{proposition}
    Sea \(H\) Hilbert e \(\{x_1, \ldots, x_n\}\) sistema ortonormal finito de \(H\).
    \begin{enumerate}[label = (\alph*)]
        \item Si \(M = [x_1, \ldots, x_n]\) e \(x \in H\), entonces \(\left\|x - \overset{n}{\sum} \langle x, x_j \rangle x_j\right\|= \text{dist}(x,M)\). 
        \item \(\forall x \in H\), se tiene \(\overset{n}{\sum} |\langle x, x_j\rangle|^2\leq \|x\|^2\).  
    \end{enumerate}
\end{proposition}

\- \vspace{0.75cm} \\  
\- \todo[gris]{ 
\[J_k := \left\{j: |\langle x, x_j\rangle| > \frac{1}{k} \right\}\]
Vea que es finito usando el item (b) de la proposición anterior
}
\begin{lemma}
    Sean \(H\) Hilbert e \(S = \{x_i\}\) sistema ortonormal de \(H\). Entonces, \(\forall x \in H\setminus \{0\}\) el conjunto \(J = \{j : \langle x, x_j\rangle \neq 0\}\) es finito o contable. 
\end{lemma}
\- \vspace{-0.09in} \\
\- \todo[gris]{Suponga que \(J\) es infinito contable, reordene los sumandos } \todo[gris]{Use el ítem (b) y haga \(n\to \infty\) de las sumas parciales }
\begin{theorem}[Desigualdad de Bessel]
    Sean \(H\) Hilbert, \(S= \{x_i\}\) un sistema ortonormal de \(H\) e \(J\) como en el lemma anterior. Entonces, \(\forall x\in H\), 
    \[\sum |\langle x, x_j \rangle |^2\leq \|x\|^2.\] 
\end{theorem}

\begin{definition}
    Sean \(E\) espacio normado e \((x_n) \subset E \). La serie \(\sum x_n < \infty \) sii 
    \[\sum^N x_j = S_N\to x \in E \ \ \sim \ \ \sum x_n = x.\]  
    Es \emph{incondicionalmente convegente} si para cualquier reordenamiento \(\sum x_{\sigma(n)}<\infty\). 
\end{definition}
\- \todo[gris]{Tome \(\varphi \in E^{'}\), y considere la serie de las imagenes, observe la igualdad de las imagenes por \(\varphi \) y aplique el penultimo colorario de Hahn-Banach}
\begin{proposition}
    Sean \(E\) espacio normado e \(\sum x_n \in E\) incondicionalmente convergente. Entonces, para cualesquiera reordenamientos \(\sigma_1, \sigma_2\), tenemos 
    \[\sum x_{\sigma_1(n)} = \sum x_{\sigma_2(n)}. \] 
\end{proposition}

\- \todo[gris]{Tome \(\{y_j\}\) reordenación de \(J\) y defina 
\- \vspace{-0.3cm} \[S_n= \sum^n \langle x, y_j\rangle y_j\] 
\- \vspace{-0.4cm}\\ 
 Use la desigualdad de Bessel para mostrar que es Cauchy}
\begin{lemma}
    Sean \(H\) Hilbert, \(S= \{x_i\}\) un sistema ortonormal de \(H\) e \(J\) como en el último lemma. Entonces, \(\forall x\in H\), la serie 
    \[\sum \langle x, x_j\rangle x_j\]
    es incondicionalmente convergente. 
\end{lemma}

\- \todo[gris]{(a) \(\Leftrightarrow \) (b) La ida es directa, para la vuelta tome la serie, un reordenamiento y muestre que 
\- \vspace{-0.2cm}
\[\left \langle x - \sum \langle x, x_{k_j}\rangle x_{k_j}, x_j \right \rangle  =0 \]
\- \vspace{-0.4cm}} \todo[gris]{(b) \(\Rightarrow\) (c) Defina \(M = \overline{[S]}\), vea que pasa con los ortogonales y concluya que \(M\) solo puede ser todo \(H\) } \todo[gris]{(c) \(\Rightarrow\) (d) \(\Rightarrow\) (e) No admiten resumen, véase \cite[pág. 98]{botelho2025introduction}} \todo[gris]{(e) \(\Rightarrow\) (b) Tome \(x_0\in S^\perp\) e \(x= x_0 = y\) en la hipótesis}
\begin{theorem}
    Sean \(H\) Hilbert, \(S= \{x_i\}\) un sistema ortonormal de \(H\) e \(x,y\in H\), entonces los siguientes enunciados son equivalentes:
    \begin{multicols}{2}
    \begin{enumerate}[label = (\alph*)]
        \item \(x= \sum \langle x, x_j \rangle x_j\).  
        \item \(S^\perp = \{0\}\). 
        \item \(\overline{[S]} = H\).  
        \item \(\|x\|^2 = \sum |\langle x, x_j \rangle|^2\). \(\rightarrow \) Identidad de Parseval   
        \item \(\langle x, y \rangle = \sum \langle x, x_j\rangle \overline{\langle y, x_j\rangle}\).    
    \end{enumerate} 
    \end{multicols}
\end{theorem}
\- \vspace{0.5cm}
\subsection{Ortoganilización y Consecuencias}
\- \todo[gris]{Igualita a la de Álgebra Lineal \\ 
\- \vspace{-0.4cm}
\[x_{n+1} = \left(\sum \langle x_{n+1}, e_j \rangle e_j \right)+ v_{n+1} \] 
\- \vspace{-0.35cm}\\ 
Donde \(v_{n+1} \in [e_1, \ldots, e_n]^\perp\), tome  \vspace{-0.15cm}
\[e_{n+1} = \frac{x_{n+1}}{\|x_{n+1} \| }\]
\- \vspace{-0.35cm}
}
\begin{proposition}[Ortogonalización de Gram-Schmidt]
    Sea \((x_n)\) sucesión l.i. de vectores en \(E\) espacio con producto interno. Entonces, \(\exists (e_n)\) sucesión l.i. ortonormal tal que, 
    \[[x_1, x_2, \ldots, x_n] =[e_1, e_2, \ldots, e_n].\] 
\end{proposition}
\- \vspace{0.5cm} \\ 
% \begin{note}[Colorario]
%     Sea \((x_n)\) sucesión l.i. de vectores en \(E\) espacio normado. Entonces, \(\exists (e_n)\) sucesión l.i. ortonormal tal que, 
% \end{note}
\- \todo[gris]{\((\Leftarrow) \) Es inmediata de la equivalencia (c)} \todo[gris]{\((\Rightarrow )\) Tome \(D\) enumerable e denso en \(H\), este tiene base infinita que se puede ortonormalizar y sigue siendo densa}
\begin{theorem}
    Sea \(H\) Hilbert tal que \(\dim(H) = \infty\). Entonces, \(H\) es separable sii \(\exists S = \{x_j\}\)  contable, tal que \(S\) es sistema ortonomal completo de \(H\).   
\end{theorem}

%\newpage 
\- \todo[gris]{Existe un \(S= \{x_n\}\) sistema ortonormal completo contable de \(H\). Por Bessel sabemos que \((\langle x, x_n\rangle ) \in \ell_2\)} \todo[gris]{Tome \(T:H \ni x \mapsto (\langle  x, x_n\rangle ) \in \ell_2\)} \todo[gris]{Bien definida e Inyectividad, represente \(x\) en serie como en (a)} \todo[gris]{Isometría, Identidad de Parseval (d)} \todo[gris]{Sobreyectividad, para \((a_n) \in \ell_2\) plantee \(\sum a_jx_j<\infty\), luego defina \vspace{-0.3cm}
\[S_N = \overset{N}{\sum}a_jx_j.\] 
\- \vspace{-0.5cm}\\ 
Use  Pitágoras para ver que \(S_n\) es Cauchy, finalmente concluya }
\begin{theorem}[Riesz-Fischer]
    Todo espacio de Hilbert infinito-dimesional separable es isometricamente isomorfo a \(\ell_2\). 
\end{theorem}

\- \vspace{4.2cm} 
\begin{exercise}
    Si \(E\) espacio normado es isomorfo a un espacio reflexivo, entonces \(E\) es reflexivo también.   
\end{exercise}
\begin{note}[Colorario]
    Los espacios de Hilbert separables son reflexivos. 
\end{note}
\- \todo[gris]{Defina \(\mathcal{F}\) la familia de todos los sistemas ortonormales de \(H\) tales que \(\mathcal{F} \ni S_i \supseteq S_0\)} \todo[gris]{Plantee el orden contención y muestre que \(\bigcup S_i \in \mathcal{F}\) es cota superior} \todo[gris]{Use el Lemma de Zorn para garantizar que \(\exists S\) máximal de \(\mathcal{F}\) }\todo[gris]{Suponga que \(S\) no es completo y busque la contradicción (del máximal)}
\begin{theorem}
    Sean \(H\) espacio con producto interno e \(S_0\) sistema ortonormal de \(H\). Entonces, \(\exists S \supseteq S_0\) sistema ortonormal completo de \(H\).  
\end{theorem}
\- \vspace{1.7cm}
\subsection{Funcionales Lineales y Teorema de Riesz-Fréchet }

\begin{example}
    Sean \(H\) Hilbert e \(\varphi_{y_0}: H \ni x \mapsto \langle x, y_0\rangle \), entonces el funcional \(\varphi_{y_0} \in H{'}\) e \(\|\varphi_{y_0}\|= \|y_0\|\). \textcolor{gray}{\(\rightarrow\) La continuidad la da Cauchy-Schwarz, para la igualdad de normas tome \(x= \frac{y_0}{\|y_0\|}\)} 
\end{example}

\- \todo[gris]{Suponga que \(\varphi \not\equiv 0 \), luego defina \(H> M\ce = \ker (\varphi)\)} \todo[gris]{Tome \(x_0 \in M^\perp\) tal que \(\|x_0\| = 1\) e
\vspace{-0.25cm}
 \[y_0:= \overline{\varphi(x_0)}x_0\] \- \vspace{-0.5cm} } \todo[gris]{Cálcule \(\langle x, y_0\rangle \) escribiendo \(x\) como \vspace{-0.2cm}
 \[\left(x +\frac{\varphi(x)}{\varphi(x_0)}x_0\right) - \frac{\varphi(x)}{\varphi(x_0)}x_0\]
 \- \vspace{-0.35cm} \\ } \todo[gris]{Cauchy-Schwarz completa la igualdad de normas, la unicidad es quasi-directa}
\begin{theorem}[Riesz-Fréchet]
    Sean \(H\) espacio de Hilbert e \(\varphi \in H{'}\). Entonces, \(\exists !y_0 \in H \) tal que \(\varphi(x) = \langle x, y_0\rangle \), más aún \(\|\varphi\| = \|y_0\|\). 
\end{theorem}
\- \vspace{3cm} 
\begin{note}[Colorario]
    Todo \(H\) Hilbert (sobre \(\R\)) es isométricamente isomorfo a \(H{'}\). \textcolor{gray}{\(\rightarrow\) Combine el Example e Teorema previos}
\end{note}
\begin{exercise}
    Si \(H\) es Hilbert, entonces \(H \simeq H{'}\) isométricamente. \textcolor{gray}{\(\rightarrow \) pend.} 
\end{exercise}
\- \todo[gris]{Tome \(\varphi_1(x) = \langle x, y_1\rangle\) e \( \varphi_2(x)= \langle x, y_2 \rangle\in H{'}\), defina \vspace{-0.25cm}
\[\langle \varphi_1, \varphi_2\rangle  := \langle y_2, y_1\rangle\] \- \vspace{-0.5cm}} 
\begin{proposition}
    Si \(H\) es Hilbert, entonces \(H{'}\) también es Hilbert. 
\end{proposition}
%\newpage 
\- \todo[gris]{Use Riesz-Fréchet\(\times 3\) para desarmar el dual del dual y mostrar que en efecto para \(\Phi \in H{''}, \ \psi \in H{'}\),  \vspace{-0.25cm} 
\[J_H(y)(\psi) = \Phi(\psi)\] \- \vspace{-0.5cm}}
\begin{note}[Colorario]
    Todo espacio de Hilbert es reflexivo. 
\end{note}

\- \vspace{0.1cm}
\begin{definition}
    Sean \(E \text{ e }F\) Banach. Una forma bilinieal \(T:E\times F\to \K\) es 
    \begin{enumerate}[label=(\alph*)]
        \item \emph{Coerciva} si \(E= F\), \(\K= \R\) e \(\exists \beta>0\), \(\forall x\in E \) tal que \(T(x,x)\geq \beta \|x\|^2\).  
        \item \emph{Simetrica} si \(E=F\) e \(\forall x,y \in E\) se tiene \(T(x,y)= T(y,x)\). 
        \item \emph{No degenerada} si \(\forall x \in E, \forall y \in F\) se tiene que \(T(x,y) = 0 \Rightarrow\) \(x=0\) o \(y=0\). 
    \end{enumerate}
\end{definition}

\begin{example}
    Las formas bilineales \(T_{1,2}: \ell_2 \times \ell_2 \to \K \) tales que 
    \begin{itemize}
        \item \(T_1: (a,b)\mapsto \sum a_jb_j \ \ \rightarrow \)  Símetrica, coerciva, continua y no degenerada. 
        \item \(T_2: (a,b) \mapsto \sum_{} a_{2j}b_{2j}\ \ \rightarrow\) Símetrica, continua, no coerciva y degenerada.  
    \end{itemize}
\end{example}
\- \todo[gris]{\(T\) es define un producto interno en \(H\)} \todo[gris]{Vea que \((x_n)\) Cauchy en \(\|\cdot \|_T\) \(\Rightarrow\) Cauchy en \( \|\cdot\|\), y van al mismo límite} \todo[gris]{Aplique Riesz-Fréchet }
\begin{proposition}
    Sean \(H\) Hilbert sobre \(\R\) e \(T:H\times H\to \R\) forma bilineal, símetrica, continua y coerciva. Entonces, \(\forall \varphi'\in H{'}, \exists !\ x_0 \in H\) tal que \(\forall x \in H\) se tiene \(\varphi(x) = T(x,x_0)\). 
\end{proposition}
\- \vspace{0.3cm}\\ 
\- \todo[gris]{\(\forall x \in H, \ T_x: y \mapsto T(y,x) \in H{'}\), por Riesz-Fréchet \(T_x(y) = \langle y, w_x\rangle\)} \todo[gris]{Pruebe que \(A: x \mapsto w_x \in \Li(H,H)\), e use coercividad para ver que es un isomorfismo en su imagen } \todo[gris]{Vea que \(\text{Ran(A)} = G\ce \leq H\) y muestre que \(G^\perp = \{0\}\)} \todo[gris]{Aplique Riesz-Fréchet y complete el argumento, \(\exists x_0\) tal que \(\varphi(x) = \langle x, A(x_0)\rangle = \langle x, w_o\rangle = T(x,x_0)\)}
\begin{theorem}[Lax-Milgram]
    Sean \(H\) Hilbert sobre \(\R \) e \(T:H\times H\to \R\) forma bilineal continua y coerciva. Entonces, \(\forall \varphi \in H{'}, \exists ! x_0 \in H \) tal que \(\varphi(x)= T(x, x_0) \). 
\end{theorem}
\- \vspace{2.9cm}\\  
\- \todo[gris]{Defina \(A:E\to F{'}\) tal que \(A(x)(y) = T(x,y)\), inyectiva e continua } \todo[gris]{\((\Rightarrow)\) Admita la representación, aplique Teorema de la Aplicación Abierta} \todo[gris]{\((\Leftarrow )\) Suponga que \(A:E \not\twoheadrightarrow F{'}\), Hahn-Banach y la reflexividad para buscar la contradicción}
\begin{theorem}[Lax-Milgram (Versión Banach)]
    Sean \(E\) e \(F\) Banach, con \(F\) reflexivo. Sea \(T:E\times F \to \K \) forma bilineal continua no degenerada. Entonces \(\forall \varphi \in F{'}, \exists  !\ x_0 \in E\) tal que \(\varphi(y)= T(x_0, y )\) sii  \(\exists \beta >0, \forall x \in E\) tal que \(\displaystyle\sup_{\|y\|=1} |T(x,y)|\geq \beta \|x\|\). 
\end{theorem}
\- \\ 
\section{Weak Topology}

%\begin{note}
%    En adelante entendemos \((X, \tau)\) como espacio topológico. 
%\end{note}
\subsection{Topología Generada por una Familia de Funciones}

\begin{definition}%[Weak Topology]
    Sean \(X\neq \emptyset\), \((Y_i, \tau_i)\) familia de espacios topológicos e \((f_i)\) familia de funciones \(f_i: X \to Y_i\). Llamamos \emph{weak topology} (sobre \(X\)) a la generada por,    
    \[\Phi := \left\{ \bigcap f_j^{-1}(A_j): A_j\ab\subseteq Y_j\right\},\]
    a la cual denotamos en adelante \(\tau_w\), topología generada por las \((f_i)\).   
\end{definition}

\begin{definition}
    Sea \(E\) espacio normado. Denotamos \((E,\sigma(E,E{'}))= (E,\tau_w)\) al espacio \(E\) con la weak topology generada por \((\varphi_i) = E{'}\).  
\end{definition}
\begin{note}
    Si \((x_n)\subseteq E\) converge en \(\tau_w\), escribimos \(x_n \overset{w}{\longrightarrow} x\).
\end{note}
\- \todo[gris]{(a), (c) y (e) son inmediatos de resultados básicos de topología} \todo[gris]{(b) Tome \(U\ab\subseteq \tau_w\) y planteelo como intersección finita de preimágenes} \todo[gris]{(d) Recuerde que \(E{'}\) separa puntos, e \(\K\) es Hausdorff}
\begin{proposition}
    Sea \(E\) espacio normado. Entonces,  
    \begin{enumerate}[label = (\alph*)]
        \item \(\forall \varphi \in E{'}\), \(\varphi: (E,\tau_w) \to \K \) es continua. 
        %\item Si \(\Im:=\{\tau : \forall i \in I, \ f_i \text{ es continua en } (X,\tau)\} \), entonces \(\tau_w = \bigcap \Im \). 
        \item \(\forall x_0 \in E\), los conjuntos de la forma \(B_{J,\epsilon} = \{ x\in E : |\varphi_j(x) - \varphi_j(x_0)|< \epsilon\}\) forman una base de vecindades \(\mathcal{B}_{x_0}\) en \(\tau_w\). 
        %\item Sea \((x_\lambda)\subseteq X \) una red. Entonces, \(x_\lambda \to x \) en \((X, \tau) \) sii \(\forall i \in I\) se tiene \(f_i(x_\lambda )\to f_i(x)\) en \((Y_i,\tau_i)\).  
        \item \(x_n \overset{w}{\longrightarrow} x \) sii \(\forall \varphi \in E{'}\), \(\varphi(x_n )\to \varphi(x)\).  
        \item \((E, \tau_w)\) es Hausdorff. 
        \item \(f: (Z, \tau) \to (E, \tau_w)\) es continua sii \(\forall \varphi \in E{'} \), \(\varphi \circ f: Z \to \K\) es continua.
        %\item Si \(\forall i \in I\), \(Y_i\) es Hausdorff, entonces \((X,\tau_w)\) es Hausdorff sii la familia de funciones \((f_i)\) separa puntos.
    \end{enumerate}
\end{proposition}
\- \todo[gris]{\(\varphi \in E{'}\) continua \(+\) ítem (c)}
\begin{note}[Colorario]
    En \(E\) espacio normado, si \(x_n\to x \), entonces \(x_n \overset{w}{\longrightarrow} x\). 
\end{note}

\begin{example}
    Para \((e_n) \subseteq c_0\)\todo[red, noline]{\hspace{1cm}\(x_n \overset{w}{\longrightarrow} x\) \textcolor{red}{\(\nRightarrow\)} \(x_n \to x \)} tenemos \(e_n \overset{w}{\longrightarrow} 0 \) mientras en \(\tau_{\|\cdot\|}\) ni siquiera es Cauchy. \textcolor{gray}{\(\rightarrow \) Tome \(\varphi \in (c_0){'}\) e use la dualidad \(+\) ítem (c) para ver que \(\varphi(e_n) \to 0=\varphi(0)\)}%\(\exists a_n \in \ell_1\) tal que \(\varphi(b_n) = \sum a_jb_j\). \(e_n \overset{w}{\longrightarrow} 0\) }
\end{example}

\begin{proposition}
    Sean \(E\) espacio normado e \(x_n \overset{w}{\longrightarrow} x\). Entonces, \\ 
    \- \hspace{0.95cm}\todo[gris]{(a) Por el ítem (c) previo \((\varphi(x_n))\) es acotada en \( E{'} \Rightarrow (x_n)\) acotada en \(E\)} \todo[gris]{Note que \(|\varphi(x)| \leq \|\varphi\|\liminf \|x_n\|\), luego aplique colorario de H-B}
    \begin{enumerate}[label = (\alph*)]
        \item \vspace{-0.1cm}\((\|x_n\|) \) es acotada, y \(\|x\|\leq \liminf \|x_n\| \).
        \begin{exercise}[4.5.12]
            Sean \(E\) espacio normado e \(B\subseteq E \). Si \(\forall \varphi \in E{'}\), \(\varphi(B)\) es acotado, entonces \(B\) es acotado también. 
        \end{exercise}
        \- \vspace{-0.6cm} \\ 
        \- \todo[gris]{Amarre \(|\varphi_n(x_n) - \varphi(x)|\) a \(\epsilon\) usando las convergencias y el ítem previo }
        \item\vspace{-0.1cm} Si \(\varphi_n\to \varphi \in E{'}\), entonces \(\varphi_n(x_n) \to \varphi(x)\). 
    \end{enumerate} 
\end{proposition} 
\- \todo[gris]{La primera afirmación es un facto topológico, \(\nexists \ \tau \subset \tau_w\) que haga todas las \((f_i)\) continuas en \((X, \tau )\)}  \todo[gris]{No admite resumen, véase \cite[pág. 121]{botelho2025introduction}}
\begin{proposition}
    Si \(E\) es espacio normado entonces \(\tau_w \subseteq \tau_{\|\cdot\|}\), con igualdad sii \(\dim E < \infty\). 
\end{proposition}
\- 
\begin{note}[Colorario]
    Si \(E\) es espacio normado, entonces \(E{'} = (E, \tau_w){'}\). 
\end{note}

\- \todo[gris]{Tome una red \((x_\lambda, f(x_\lambda)) \to (x,y) \in X \times Y\), use la continuidad de las proyecciones y del luego la de \(f\)}
\begin{lemma}
    Sea \(f: (X,\tau) \to (Y,\tau{'})\) continua. Si \((Y, \tau{'})\) es Hausdorff entonces \(\operatorname{graf}(f)\ce\subseteq X\times Y\) con la topología producto.   
\end{lemma}

\- \todo[gris]{\((\Rightarrow)\) \(\forall \varphi \in F{'}\), \(\varphi \circ T_\sigma \in (E, \tau_w^E){'}\) } \todo[gris]{\((\Leftarrow )\) Por el lemma anterior \(\operatorname{graf}(T_\sigma)\ce\) en \((E,\tau_w^E)\times (F, \tau_w^F) \subseteq (E\times F, \tau_{\|\cdot\|})\), se sigue de Teorema del Gráfico Cerrado}
\begin{proposition}
    Sean \(E\) e \(F\) Banach. Un operador lineal \(T: E \to F\) es continuo sii \(T_\sigma:(E, \tau_w^E) \to (F,\tau_w^F)\) es continuo. 
\end{proposition}
\- \vspace{-0.3cm}
%\begin{note}
%    Vale para \(E\) normado en general pero la demostración es un poco más elaborada. 
%\end{note}
\- \vspace{0.3cm } \\ 
\- \todo[gris]{\((\subseteq )\)  Es inmediato de la "preservación" de convergencia en sucesiones} \todo[gris]{\((\supseteq )\) Sea \(x_0 \in \overline{K}^{ \ \tau_{w}} \setminus \overline{K}^{ \ \tau_{\|\cdot\|}}\), consiga aplicar versión geométrica H-B (estricta) y busque la contradicción} %\todo[gris]{\((\supseteq)\) \((\K = \C)\) Análogo}
\begin{theorem}[Mazur]
    Sean \(E\) espacio normado e \(K\subseteq E \) convexo. Entonces \(\overline{K}^{ \ \tau_{\|\cdot\|}} = \overline{K}^{ \ \tau_w}\). En particular, \(K\ce \subseteq (E, \tau_{\|\cdot\|})\) sii \(K\ce \subseteq (E, \tau_{w})\). 
    \begin{exercise}[1.8.19]
        \(K\subseteq E\) es convexo \(\Rightarrow \overline{K}\) convexo.  
    \end{exercise}
\end{theorem}

\- \todo[gris]{No admite resumén, véase \cite[pág. 124]{botelho2025introduction}}
\begin{theorem}[Schur]
    Sea \((x_n) \subseteq \ell_1\). Entonces \(x_n \to x \) sii \(x_n \overset{w}{\longrightarrow} x\). 
\end{theorem}

\subsection{Weak-Star Topology}

\begin{definition}
    Sea \(E\) espacio normado la \emph{weak-star topology} definida en \(E{'}\) y a la cual denotamos \(\sigma(E{'}, E) = (E{'},\tau_{w^*} )\) es la generada por la familia \(J_E(E)\subseteq E{''}\). 
    %\[\forall x\in E, \ \forall \varphi \in E{'}, \ J_E(x):  \varphi \mapsto \varphi(x) \in \K. \] 
\end{definition}

\begin{note}
    Análogamente, cuando \((\varphi_n) \subseteq E{'}\) converge en \(\tau_{w^*}\), escribimos \(\varphi_n \overset{w^*}{\longrightarrow}\varphi \). 
\end{note}

\- \todo[gris]{(a), (c) y (e) de nuevo son consecuencias de resultados topológicos}
\todo[gris]{(b) Es una adaptación de su análogo}
\todo[gris]{(d) \(J_E(E)\) separa puntos también, veáse Exercise 4.5.11}
\begin{proposition}
    Sea \(E\) espacio normado. Entonces, %\(\tau_{w^*}\),  
    \begin{enumerate}[label = (\alph*)]
        \item \(\forall x \in E\), \(J_E(x): (E{'},\tau_{w^*}) \to \K \) es continua. 
        %\item Si \(\Im:=\{\tau : \forall i \in I, \ f_i \text{ es continua en } (X,\tau)\} \), entonces \(\tau_w = \bigcap \Im \). 
        \item \(\forall \varphi_0 \in E{'}\), los conjuntos de la forma \(B_{J,\epsilon} = \{\varphi \in E{'} : |\varphi_j(x_j) - \varphi_0(x_j)|< \epsilon\}\) forman una base de vecindades \(\mathcal{B}_{\varphi_0}\) en \(\tau_{w^*}\). 
        %\item Sea \((x_\lambda)\subseteq X \) una red. Entonces, \(x_\lambda \to x \) en \((X, \tau) \) sii \(\forall i \in I\) se tiene \(f_i(x_\lambda )\to f_i(x)\) en \((Y_i,\tau_i)\).  
        \item \(\varphi_n \overset{w^*}{\longrightarrow} \varphi \) sii \(\forall x \in E\), \(\varphi_n(x )\to \varphi(x)\).  
        \item \((E{'}, \tau_{w^*})\) es Hausdorff. 
        \item {\(f: (Z, \tau) \to (E{'}, \tau_{w^*})\)} es continua sii {\small\(\forall x \in E, \ J_E(x) \circ f: Z \to \K\)} es continua. 
        %\item Si \(\forall i \in I\), \(Y_i\) es Hausdorff, entonces \((X,\tau_w)\) es Hausdorff sii la familia de funciones \((f_i)\) separa puntos.
    \end{enumerate}
\end{proposition}
\begin{example}
    Sea \((e_n)\subseteq \ell_1 = (c_0){{'}}\). Dada \(x = (x_n) \in c_0\), entonces \(e_n ( x) = x_n \to 0 \), luego por el ítem (c) \(e_n \overset{w^*}{\longrightarrow } 0\). 
\end{example}
\- \todo[gris]{Todos los ítems son adaptaciones de resultados anteriores}
\begin{proposition}
    Sea \(E\) espacio normado. Entonces, 
    \begin{enumerate}[label = (\alph*)]
        \item Si \(\varphi_n \overset{w}{\longrightarrow } \varphi\) entonces \(\varphi_n \overset{w^*}{\longrightarrow } \varphi\). 
        \item Si \(E\) es Banach e \(\varphi_n \overset{w^*}{\longrightarrow } \varphi \) entonces \(\left(\|\varphi_n\|\right)\) es acotada y \(\|\varphi\| \leq \liminf \|\varphi_n\|\). 
        \item Si \(E\) es Banach, \(\varphi_n\overset{w^*}{\longrightarrow } \varphi\) e \(x_n \to x \in E \), entonces \(\varphi_n(x_n) \to \varphi(x) \in \K\). 
    \end{enumerate}
\end{proposition}

\- \todo[gris]{Tome \(T: V \ni x \mapsto (\varphi_j(x))\in \K^n\)}
\- \todo[gris]{Haga \(U: T(V) \ni x\to \varphi(x) \in \K \), vea que está bien definida, planteé una extensión y concluya }
\begin{lemma}
    Sean \(V\) espacio vectorial e \(\varphi_j \in V{'}\) tales que \(\bigcap \ker \varphi_j \subseteq \ker \varphi\), entonces \(\exists a_j\in \K \) tales que \(\varphi = \sum a_j \varphi_j\). 
\end{lemma}
\- \vspace{0.2cm}\\ 
\- \todo[gris]{Tome \(f \in (E{'}, \tau_{w^*})\), entonces \(f(0)= 0\), luego \(\exists B_{J, \epsilon }\) donde \(|f(\varphi)|<1\) \vspace{-0.2cm}
\[B_{J, \epsilon } = \{\varphi \in E{'}: |\varphi(x_j)|< \epsilon \}\] \- \vspace{-0.5cm}} 
\todo[gris]{Suponga que \(\forall j, \ J_E(x_j)(\varphi) = 0 \), apunte a mostrar que \(f(\varphi )= 0\), \vspace{-0.2cm}
\[\bigcap \ker (J_E(x_j)) \subseteq \ker f \]\- \vspace{-0.5cm} \\ 
Aplique el lemma y concluya }
\begin{proposition}
    Si \(E\) es espacio normado entonces \((E{'},\tau_{w^*}){'} = J_E(E)\). 
\end{proposition}

\begin{note}[Colorario]
    Si \(E\) es espacio normado, entonces \((E{''},\tau_{w^*}){'} = J_{E{'}}(E{'})\)
\end{note}
\- \vspace{1.8cm}

\- \todo[gris]{Recuerde que \(J_E(E)\subseteq E{''}\)}
\begin{proposition}
    Sea \(E\) espacio normado. Entonces, \((E{'},\tau_{w^*})\subseteq (E{'}, \tau_w)\), más aún, coinciden sii \(E\) es reflexivo. 
\end{proposition}

\- \todo[gris]{No admite resumen, veáse \cite[pág. 129]{botelho2025introduction}}
\begin{theorem}[Banach-Alaoglu-Bourbaki]
    Sea \(E\) espacio normado, entonces la bola \(B_{E{'}}\) es compacta en \(\subseteq (E{'}, \tau_{w^*})\). 
\end{theorem}

%%%%%%%%%%%%%%%%%%%%%%%%%% REFERÊNCIAS %%%%%%%%%%%%%%%%%%%%%%%%%%%%%%%%

\begin{thebibliography}{9}

\bibitem{botelho2025introduction}
Botelho, Geraldo and Pellegrino, Daniel and Teixeira, Eduardo. \textit{Introduction to Functional Analysis}, Springer, 2025.

\end{thebibliography}

%%%%%%%%%%%%%%%%%%%%%%%%%% APÊNDICES %%%%%%%%%%%%%%%%%%%%%%%%%%%%%%%%

\end{document}
